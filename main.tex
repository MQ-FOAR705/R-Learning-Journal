\documentclass{article}
\usepackage[utf8]{inputenc}

\title{R-Learning-Journal}
\author{Jeremy Amin}
\date{October 2019}

\begin{document}

\maketitle

\section{Before we Start - 30 OCT 2019}

\begin{itemize}
    \item `R' refers to both the programming language as well as the software that interprets the scripts written using it.
    \item R is preferable for a number of reasons. 1. It relies on written commands, not a mouse. (Written commands may not be as user friendly for the typical GUI user but replicability is far more reliable. 2. Working with scripts forces the user to be more intimately connected with their analysis or whatever other task they wish to conduct with R. This facilitates learning and grants more control to the user. 3. R is interdisciplinary and extensible. It allows for the application of more than one discipline's methods of analysis for data analysis. 4. R allows for the production of high-quality graphics. 5. R has a community of users and people with know-how in case a person needs help. 6. R is free, open-source and cross platform.
    \item Opened a New project and created a new file within the project.
    \item Default layout for RStudio is: Top Left - Source: pane for my scripts and documents
Bottom Left - Console: pane to show me what R would look and be like without RStudio
Top Right - Enviornment/History: pane to show me what I have done.
Bottom Right - Files and more: pane to allow me to see the contents of the project/working directory
\item Created folders from Console using commands: dir.create("data"),
dir.create("data\textunderscore output"), and
dir.create("fig\textunderscore output")
\item I installed the `tidyverse' package from the drop-down menu. If I want to download packages in the future I can also type the following command into the Console: install.packages()

\end{itemize}

\section{Introduction to R - 4 NOV 2019}

\begin{itemize}
    \item To assign values to objects the operator is <-
    \item E.G. To make x = 10 the command is x <- 10
    \item In the first exercise the correct answer is 6.175 because although the value of area\textunderscore hectares was changed to 50 this is written after the line to multiply it by 2.47 to convert it into acres.
    \item I assigned values for length, width, and area. Below are the lines I used.
    \item > length <- 5
> width <- 10
> area <- length * width
> area
[1] 50
\item The value of area did not change when I changed the values of width and length. This is for the same reason as the previous exercise. Once I reassigned area to be length multiplied by width the new values of length and width were used and a new value of area was made.
\item 
\end{itemize}

\end{document}
